%!TEX root = /Users/tds/thesisLatek/template.tex
%Developed using Sublime Text 3 with LaTexTools plugin
%SUPER+b for compiling latex to pdf
%Don't forget to use SUPER+R to jump between top level commands!
%%%%%%%%%%%%%%%%%%%%%%%%%%%%%%%%%%%%%%%%%%%%%%%%%%%%%%%%%%%%%%%%%%%%
%% chapter2.tex
%% UNL thesis document file
%%
%% Chapter with the template manual
%%%%%%%%%%%%%%%%%%%%%%%%%%%%%%%%%%%%%%%%%%%%%%%%%%%%%%%%%%%%%%%%%%%%
\chapter{State of the Art}
\label{cha:state_of_the_art}
% ================
% = Introduction =
% ================

Bla bla bla dizer que há metodos convencionais tipo ESA e dizer que há malta já a usar machine learning.
explicar o que é assim de alto a Análise da assinatura elétrica 
Explicar as diferenças entre as abordagens por ESA e por machine learning, só que breve


\section{Machine Current Signature Analysis} % (fold)
\label{sec:mcsa}

\section{Extended Park's Vector Approach} % (fold)
\label{sec:epva}

Stator Winding Fault Diagnosis in Three-Phase Synchronous and Asynchronous Motors, by the Extended Park’s Vector Approach


\section{Park-Hilbert Method} % (fold)
\label{sec:park_hilbert_method}

\section{Symmetrical Components under Stator Faults} % (fold)
\label{sec:symmetrical_components}

\section{Negative Sequence Current Compensation} % (fold)
\label{sec:negative_sequence_current_compensation}

\section{Zero Sequence Components} % (fold)
\label{sec:zero_sequence_components}

\section{Using Neural Networks} % (fold)
\label{sec:using_nn}

Park’s Vector Approach to detect an inter turn stator fault in a doubly fed induction machine by a neural network 

Recent Advances in Modeling and Online Detection of Stator Interturn Faults in Electrical Motors

\section{Using Support Vector Machines} % (fold)
\label{sec:using_svm}

\section{Summary} % (fold)
\label{sec:related_work_summary}
