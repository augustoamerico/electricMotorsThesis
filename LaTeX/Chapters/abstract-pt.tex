%%%%%%%%%%%%%%%%%%%%%%%%%%%%%%%%%%%%%%%%%%%%%%%%%%%%%%%%%%%%%%%%%%%%
%% abstrac-pt.tex
%% UNL thesis document file
%%
%% Abstract in Portuguese
%%%%%%%%%%%%%%%%%%%%%%%%%%%%%%%%%%%%%%%%%%%%%%%%%%%%%%%%%%%%%%%%%%%%
Os motores elétricos fazem parte da grande maioria dos equipamentos industriais, sendo responsáveis por 30\%-40\% da energia elétrica consumida mundialmente.
Apesar da probabilidade de avaria ser bastante baixa, estes motores são cruciais para a indústria e, como tal, há preocupações constantes com a sua fiabilidade e disponibilidade.
Dado que uma falha inesperada do motor pode resultar num grande prejuízo, o investimento em práticas de manutenção  que melhorem tanto a disponibilidade como a eficiência dos motores é muito atractivo para a indústria.

Neste contexto, esta dissertação propõe abordar falhas de curto circuito. Este tipo de falhas exige uma deteção ainda num estado inicial e uma reação rápida à mesma, sendo necessário a deteção das mesmas em tempo real.

Para tal, técnicas de aprendizagem automática e de tratamento digital de sinal serão estudadas para apresentar um modelo capaz de identificar falhas ainda num estado inicial em tempo real.

% Palavras-chave do resumo em Português
\begin{keywords}
Motor elétrico, Motor de indução com rotor em gaiola de esquilo, deteção de falhas de curto-circuito, aprendizagem automática \ldots
\end{keywords}
% to add an extra black line
