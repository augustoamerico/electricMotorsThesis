%!TEX root = ../template.tex
%Developed using Sublime Text 3 with LaTexTools plugin
%SUPER+b for compiling latex to pdf
%Don't forget to use SUPER+R to jump between top level commands!
%F6 to spell check
%%%%%%%%%%%%%%%%%%%%%%%%%%%%%%%%%%%%%%%%%%%%%%%%%%%%%%%%%%%%%%%%%%%%
%% chapter4.tex
%% UNL thesis document file
%%
%% Chapter with the info about the available data
%%%%%%%%%%%%%%%%%%%%%%%%%%%%%%%%%%%%%%%%%%%%%%%%%%%%%%%%%%%%%%%%%%%%
\chapter{Proposed Approach}
\label{cha:available_data}
% ================
% = Introduction =
% ================

For this dissertation's work, it's proposed a CRISP-DM methodology to approach the problem. An explanation about this methodology is presented in section \ref{sec:crispdm}, following a suggested working plan for this dissertation in section \ref{sec:work_plan}. In section \ref{sec:available_data} the available data is presented and its quality is discussed, referring that an additional dataset is expected.

As off the time this document is being written, the Business Understanding phase has already finished and the current focus is on the Data Understanding and the Data Preparation.

explicar o crisp-dm

colocar aqui o primeiro nivel de cada tarefa

na parte de utilização das tecnicas, ser mencionado de forma suave o que é cada técnica

available data


\section{\Acrlong{crispdm}} % (fold)
\label{sec:crispdm}

\section{Work Plan} % (fold)
\label{sec:work_plan}

\section{Available Data}
\label{sec:available_data}
