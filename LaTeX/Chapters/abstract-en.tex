%%%%%%%%%%%%%%%%%%%%%%%%%%%%%%%%%%%%%%%%%%%%%%%%%%%%%%%%%%%%%%%%%%%%
%% abstrac-en.tex
%% UNL thesis document file
%%
%% Abstract in English
%%%%%%%%%%%%%%%%%%%%%%%%%%%%%%%%%%%%%%%%%%%%%%%%%%%%%%%%%%%%%%%%%%%%
Electric motors are a great majority of industrial equipments, and are responsible for 30\%-40\% of the electrical energy's consumption. 
Although the probability of breakdowns of electric motors is very low, these motors are critical to the Industry and therefore there are constant concerns with their high availability and reliability.
Since an unexpected failure can have high costs associated, the investment in maintenance practices that improve the efficiency and availability of electrical motors is very attractive to the Industry.

In this context, this dissertation proposes to approach a specific kind of failure, which is referred as turn insulation faults. 
This kind of failures demands for early detection and a fast reaction time to the failure detection, being necessary the online detection of these faults.

To do so, machine learning and digital signal processing techniques will be study to present a model capable of identify incipient faults in real time.

% Palavras-chave do resumo em Inglês
\begin{keywords}
\acrfull{em}, \acrfull{ims}, turn insulation fault detection, short-circuit fault detection, machine learning \ldots
\end{keywords} 
